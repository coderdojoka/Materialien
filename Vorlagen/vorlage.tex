% Die folgenden zwei Befehle binden die CoderDojo-Styles ein.
% ACHTUNG: der Pfad ist relativ und muss entsprechend der Ordnerstruktur angepasst werden!
% Beispiel: - der Vorlagen-Ordner liegt zwei Ebenen tiefer => ../../Vorlagen
%			
\newcommand*{\VorlagenPfad}{.}	% wir sind im gleichen Verzeichnis, wie die Style-Datei => '.'
\documentclass{\VorlagenPfad/coderdojokatext}
% Titel als Kommando für den Header/Footer definieren
\renewcommand{\Titel}{Der Titel}

\usepackage{enumitem}
\usepackage[autostyle,german=swiss]{csquotes}
\begin{document}


% Titel anzeigen und um den Header/Footer zu generieren
\begin{center}
	{\huge \Titel}
\end{center}

\section{Bubberfisch}
Bubberfisch bubberte durch Bubberfischhausen.


\enquote{Unteilbar}

\begin{enumerate}
	\item Bubberei
	\item LLisdfj 
	\item dfjlsdlfj 
\end{enumerate}

\section{Bubber-Code}
Quellcode kann aus einer Datei eingelsen werden...
\inputminted[firstline=5, lastline=8,linenos, breaklines, frame=single, framesep=10pt,framerule=1pt]{python}{../Python/Beispiele/zahlen_raten.py}

oder auch direkt hier stehen

\begin{pythoncode}
print("Hallo Welt")
\end{pythoncode}

\end{document}