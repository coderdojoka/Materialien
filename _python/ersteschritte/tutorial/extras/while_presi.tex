\newcommand*{\VorlagenPfad}{../../../../Vorlagen}%
\documentclass{\VorlagenPfad/coderdojokabeamer}

\title{Schleifen}
\author{CoderDojo Karlsruhe}
\date{Mai 7, 2015}

\newcommand{\code}[1]{{\ttfamily #1}}

\begin{document}

\maketitle

\section{Wiederholung}

\begin{frame}{Bedingungen} 
Eine Bedingung ist eine Aussage, die nur zu Wahr(\code{True}) oder Falsch(\code{False}) ausgewertet werden kann.

\end{frame}

\begin{frame}[fragile]{if-Abfrage}
	\begin{minted}{python}
# wert = ?
if wert < 10:
	print("Kleiner 10.")
else:
	print("Größer oder gleich 10.")

print("Nach if-Abfrage.")
	\end{minted}
\end{frame}

\begin{frame}[fragile]{if-Abfrage}
	\begin{minted}{python}
# wert = ?
if wert < 10:
	print("Kleiner 10.")
else:
	print("Größer oder gleich 10.")
		
print("Nach if-Abfrage.")
	\end{minted}
Welche Anweisung ist hier die Bedingung?\\

\end{frame}



\begin{frame}[fragile]{if-Abfrage}
	\begin{minted}{python}
# wert = ?
if wert < 10:
	print("Kleiner 10.")
else:
	print("Größer oder gleich 10.")
	
print("Nach if-Abfrage.")
\end{minted}

Was wird ausgegeben, wenn \code{wert = 12} ist.
\end{frame}






\section{while-Schleife}

\begin{frame}{Aufgabe}
Du hast die Aufgabe die Zahlen von \code{1} bis \code{10} auszugeben.
\end{frame}

\begin{frame}[fragile]{Erste Version}
	\begin{minted}{python}
# alle Zahlen einfach nach einander ausgeben
print("1")
print("2")
print("3")
print("4")
print("5")
print("6")
print("7")
print("8")
print("9")
print("10")
	\end{minted}
\end{frame}


\begin{frame}[fragile]{Zweite Version}
	\begin{minted}{python}
zaehler = 1 # Zählervariable
while zaehler <= 10: # wiederhole solange zaehler <= 10
	print(zaehler) # gibt den aktuellen Wert aus
	zaehler = zaehler + 1 # erhöht den Zähler um 1
	# wiederhole
	\end{minted}
\end{frame}

\begin{frame}{while-Schleife}
Eine \code{while}-Schleife wiederholt solange den Schleifenkörper (die eingerückten Zeilen), wie die Bedingung wahr ist.
\end{frame}
	
\begin{frame}{Analyse}
Um zu sehen was genau in der \code{while}-Schleife passiert gehen wir alle Durchläufe einzeln durch.
\end{frame}


% Erster Durchlauf 

\begin{frame}[fragile]{Durchlauf \#1}
	\begin{minted}{python}
zaehler = 1
while zaehler <= 10: # 1 <= 10, d.h. die Bedingung ist wahr
	print(zaehler)
	zaehler = zaehler + 1
	\end{minted}
\end{frame}

\begin{frame}[fragile]{Durchlauf \#1}
	\begin{minted}{python}
zaehler = 1
while zaehler <= 10:
	print(zaehler) # gibt "1" aus
	zaehler = zaehler + 1
	\end{minted}
\end{frame}

\begin{frame}[fragile]{Durchlauf \#1}
	\begin{minted}{python}
zaehler = 1
while zaehler <= 10:
	print(zaehler)
	zaehler = zaehler + 1 # zaehler = 1 + 1 = 2
	\end{minted}
\end{frame}


% Zweiter Durchlauf

\begin{frame}[fragile]{Durchlauf \#2}
	\begin{minted}{python}
zaehler = 1
while zaehler <= 10: # 2 <= 10, d.h. die Bedingung ist wahr
	print(zaehler)
	zaehler = zaehler + 1
	\end{minted}
\end{frame}

\begin{frame}[fragile]{Durchlauf \#2}
	\begin{minted}{python}
zaehler = 1
while zaehler <= 10:
	print(zaehler) # gibt "2" aus
	zaehler = zaehler + 1
	\end{minted}
\end{frame}

\begin{frame}[fragile]{Durchlauf \#2}
	\begin{minted}{python}
zaehler = 1
while zaehler <= 10:
	print(zaehler)
	zaehler = zaehler + 1 # zaehler = 2 + 1 = 3
	\end{minted}
\end{frame}

% Dritter Durchlauf

\begin{frame}[fragile]{Durchlauf \#3}
	\begin{minted}{python}
zaehler = 1
while zaehler <= 10: # 3 <= 10, d.h. die Bedingung ist wahr
	print(zaehler)
	zaehler = zaehler + 1
	\end{minted}
\end{frame}


\begin{frame}[fragile]{Durchlauf \#3}
	\begin{minted}{python}
zaehler = 1
while zaehler <= 10:
	print(zaehler) # gibt "3" aus
	zaehler = zaehler + 1
	\end{minted}
\end{frame}

\begin{frame}[fragile]{Durchlauf \#3}
	\begin{minted}{python}
zaehler = 1
while zaehler <= 10:
	print(zaehler)
	zaehler = zaehler + 1 # zaehler = 3 + 1 = 4
	\end{minted}
\end{frame}


% Vierter Durchlauf

\begin{frame}[fragile]{Durchlauf \#4}
	\begin{minted}{python}
zaehler = 1
while zaehler <= 10: # 4 <= 10, d.h. die Bedingung ist wahr
	print(zaehler)
	zaehler = zaehler + 1
	\end{minted}
\end{frame}


\begin{frame}[fragile]{Durchlauf \#4}
	\begin{minted}{python}
zaehler = 1
while zaehler <= 10:
	print(zaehler) # gibt "4" aus
	zaehler = zaehler + 1
	\end{minted}
\end{frame}


\begin{frame}[fragile]{Durchlauf \#4}
	\begin{minted}{python}
zaehler = 1
while zaehler <= 10:
	print(zaehler)
	zaehler = zaehler + 1 # zaehler = 4 + 1 = 5
	\end{minted}
\end{frame}


% Fünfter Durchlauf
\begin{frame}[fragile]{Durchlauf \#5}
	\begin{minted}{python}
zaehler = 1
while zaehler <= 10: # 5 <= 10, d.h. die Bedingung ist wahr
	print(zaehler)
	zaehler = zaehler + 1
	\end{minted}
\end{frame}

\begin{frame}[fragile]{Durchlauf \#5}
	\begin{minted}{python}
zaehler = 1
while zaehler <= 10:
	print(zaehler) # gibt "5" aus
	zaehler = zaehler + 1
	\end{minted}
\end{frame}

\begin{frame}[fragile]{Durchlauf \#5}
	\begin{minted}{python}
zaehler = 1
while zaehler <= 10:
	print(zaehler)
	zaehler = zaehler + 1 # zaehler = 5 + 1 = 6
	\end{minted}
\end{frame}



% Sechster Durchlauf
\begin{frame}[fragile]{Durchlauf \#6}
	\begin{minted}{python}
zaehler = 1
while zaehler <= 10: # 6 <= 10, d.h. die Bedingung ist wahr
	print(zaehler)
	zaehler = zaehler + 1
	\end{minted}
\end{frame}

\begin{frame}[fragile]{Durchlauf \#6}
	\begin{minted}{python}
zaehler = 1
while zaehler <= 10:
	print(zaehler) # gibt "6" aus
	zaehler = zaehler + 1
\end{minted}
\end{frame}

\begin{frame}[fragile]{Durchlauf \#6}
	\begin{minted}{python}
zaehler = 1
while zaehler <= 10:
	print(zaehler)
	zaehler = zaehler + 1 # zaehler = 6 + 1 = 7
\end{minted}
\end{frame}


% Siebter Durchlauf
\begin{frame}[fragile]{Durchlauf \#7}
	\begin{minted}{python}
zaehler = 1
while zaehler <= 10: # 7 <= 10, d.h. die Bedingung ist wahr
	print(zaehler) 
	zaehler = zaehler + 1 
	\end{minted}
\end{frame}

\begin{frame}[fragile]{Durchlauf \#7}
	\begin{minted}{python}
zaehler = 1
while zaehler <= 10: 
	print(zaehler) # gibt "7" aus
	zaehler = zaehler + 1 
	\end{minted}
\end{frame}

\begin{frame}[fragile]{Durchlauf \#7}
	\begin{minted}{python}
zaehler = 1
while zaehler <= 10: 
	print(zaehler)
	zaehler = zaehler + 1 # zaehler = 7 + 1 = 8
	\end{minted}
\end{frame}



% Achter Durchlauf
\begin{frame}[fragile]{Durchlauf \#8}
	\begin{minted}{python}
zaehler = 1
while zaehler <= 10: # 8 <= 10, d.h. die Bedingung ist wahr
	print(zaehler)
	zaehler = zaehler + 1
	\end{minted}
\end{frame}

\begin{frame}[fragile]{Durchlauf \#8}
	\begin{minted}{python}
zaehler = 1
while zaehler <= 10:
	print(zaehler) # gibt "8" aus
	zaehler = zaehler + 1
	\end{minted}
\end{frame}

\begin{frame}[fragile]{Durchlauf \#8}
	\begin{minted}{python}
zaehler = 1
while zaehler <= 10:
	print(zaehler)
	zaehler = zaehler + 1 # zaehler = 8 + 1 = 9
	\end{minted}
\end{frame}




% Neunter Durchlauf
\begin{frame}[fragile]{Durchlauf \#9}
	\begin{minted}{python}
zaehler = 1
while zaehler <= 10: # 9 <= 10, d.h. die Bedingung ist wahr
	print(zaehler)
	zaehler = zaehler + 1
	\end{minted}
\end{frame}

\begin{frame}[fragile]{Durchlauf \#9}
	\begin{minted}{python}
zaehler = 1
while zaehler <= 10:
	print(zaehler) # gibt "9" aus
	zaehler = zaehler + 1
	\end{minted}
\end{frame}

\begin{frame}[fragile]{Durchlauf \#9}
	\begin{minted}{python}
zaehler = 1
while zaehler <= 10:
	print(zaehler)
	zaehler = zaehler + 1 # zaehler = 9 + 1 = 10
	\end{minted}
\end{frame}



% Zehnter Durchlauf
\begin{frame}[fragile]{Durchlauf \#10}
	\begin{minted}{python}
zaehler = 1
while zaehler <= 10: #10 <= 10, d.h. die Bedingung ist wahr
	print(zaehler)
	zaehler = zaehler + 1
	\end{minted}
\end{frame}

\begin{frame}[fragile]{Durchlauf \#10}
	\begin{minted}{python}
zaehler = 1
while zaehler <= 10:
	print(zaehler) # gibt "10" aus
	zaehler = zaehler + 1
	\end{minted}
\end{frame}

\begin{frame}[fragile]{Durchlauf \#10}
	\begin{minted}{python}
zaehler = 1
while zaehler <= 10:
	print(zaehler)
	zaehler = zaehler + 1 # zaehler = 10 + 1 = 11
	\end{minted}
\end{frame}



% Elfter Durchlauf
\begin{frame}[fragile]{Durchlauf \#11}
	\begin{minted}{python}
zaehler = 1
while zaehler <= 10: # 11 <= 10, die Bedingung ist FALSCH!
	print(zaehler) 
	zaehler = zaehler + 1 
	
	\end{minted}
\end{frame}

\begin{frame}[fragile]{Durchlauf \#11}
	\begin{minted}{python}
zaehler = 1
while zaehler <= 10:
	print(zaehler) 
	zaehler = zaehler + 1 
		
# Hier gehts weiter (nicht eingrückt)
	\end{minted}
\end{frame}
\end{document}
