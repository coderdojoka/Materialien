%% Author: Mark weinreuter
\newcommand*{\VorlagenPfad}{../../../../Vorlagen}%
\documentclass{\VorlagenPfad/coderdojokatext}

% Titel, Kategorie müssen als Kommando für den Header/Footer definiert werden.
\newcommand{\Titel}{while- und for-Schleifen}
\newcommand{\Kategorie}{Basics:\space} 

\begin{document}

\section{Ein Fenster}
In diesem Abschnitt wirst du ein Fenster in Python erzeugen. Dazu musst du zun"achst die folgende Zeile in ein leeres Textdokument schreiben:

\inputminted[linenos,firstline=1,lastline=1]{python}{../../../Beispiele/fenster_beispiel.py}

Damit signalisierst du deinem Computer, dass du in deinem Programm Fenster verwenden m"ochtest. Er sucht dann alle Informationen, die daf"ur ben"otigt werden.\par
Anschlie"send erzeuge das Fenster durch folgenden Befehl:

\begin{pythoncode}
fenster = Tk()
fenster.mainloop()
\end{pythoncode}
	
Die erste Zeile erzeugt ein neues Objekt namens \code{fenster}. Es ist vom Typ \code{Tk}. Falls dir das an diesem Punkt noch nicht einleuchtet, lass dich ermutigen und programmiere einfach weiter. Falls du es dennoch ganz genau wissen m"ochtes, frage doch einen Tutor nach "`Objekten"'.
Die zweite Zeile zeigt das Fenster einfach an.
Jetzt kannst du das Programm ausf"uhren und wirst mit deinem ersten eigenen Python-Fenster belohnt. Geschafft.\par
Es sieht noch etwas kahl aus. Verziere es mit etwas Text:

\inputminted[linenos,firstline=5,lastline=6]{python}{../../../Beispiele/fenster_beispiel.py}

Achte darauf, dass die Zeile \code{fenster.mainloop()} unbedingt am \textbf{Ende} deines Textdokuments steht. Dies wird w"ahrend der gesamten Anleitung so bleiben. F"uhre dein Programm aus und du erh"altst ein (sehr kleines) Fenster mit deinem Text. Trommelwirbel!\par
Schauen wir uns die folgende Zeile etwas genauer an:
\begin{pythoncode}
text = Label(master=fenster, text = "Bubberfisch")
\end{pythoncode}
Es wird ein "`Label"', also ein Textfeld, mit dem Namen \code{text} erzeugt. Du musst deinem Computer nat"urlich mitteilen, wohin der Text geschrieben werden soll. Dies geschieht durch \code{master=fenster}. (Du erinnerst dich nat"urlich daran, dass \code{fenster} der Name deines Fensters ist!) Der Text selbst wird als \code{text= "Bubberfisch"} angegeben.\par
Spiele ein wenig mit dieser Zeile und schreibe verschiedene Nachrichten auf ein Fenster.

\subsection{Das komplette Beispiel}

\inputminted[linenos]{python}{../../../Beispiele/fenster_beispiel.py}

\end{document}