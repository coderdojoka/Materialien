\newcommand*{\VorlagenPfad}{../../Vorlagen/}	% wir sind im gleichen Verzeichnis, wie die Style-Datei => '.'
\documentclass{\VorlagenPfad/coderdojokatext}

\usepackage{hyperref}


\begin{document}

\begin{center}
{\huge CoderDojo Karlsruhe} \\ \ \\ \large{Scratch-Tutorial: Speicher, Steuerung, Fühlen und Operatoren}
\end{center}


% Scratch Farben
\definecolor{Bew}{RGB}{74, 108, 212}
\definecolor{Aus}{RGB}{138, 85, 215}
\definecolor{Kla}{RGB}{187, 66, 195}
\definecolor{Mal}{RGB}{14, 154, 108}
\definecolor{Spe}{RGB}{238, 125, 22}
\definecolor{Ere}{RGB}{200, 131, 48}
\definecolor{Ste}{RGB}{225, 169, 26}
\definecolor{Fue}{RGB}{44, 165, 226}
\definecolor{Ope}{RGB}{92, 183, 18}
\definecolor{Wei}{RGB}{99, 45, 153}

\newcommand{\Block}[2]{\textcolor{#1}{'#2'}}
\newcommand{\SFarbe}[2]{\textcolor{#1}{#2}}
\newcommand{\KBew}{\textcolor{Bew}{Bewegung}}
\newcommand{\KAus}{\textcolor{Aus}{Aussehen}}
\newcommand{\KKla}{\textcolor{Kla}{Klang}}
\newcommand{\KMal}{\textcolor{Mal}{Malstift}}
\newcommand{\KSpe}{\textcolor{Spe}{Speicher}}
\newcommand{\KEre}{\textcolor{Ere}{Ereignisse}}
\newcommand{\KSte}{\textcolor{Ste}{Steuerung}}
\newcommand{\KFue}{\textcolor{Fue}{Fühlen}}
\newcommand{\KOpe}{\textcolor{Ope}{Operatoren}}
\newcommand{\KWei}{\textcolor{Wei}{Weitere Blöcke}}

% \newcommand*{\color_bew}[1]{\textcolor{Bew}{$1} }


\section{Installation und Einstellungen}
Falls noch nicht geschehen lade dir den Scratch 2.0 Offline Editor herunter (frage dazu bitte einen der Mentoren) oder verwende die Online Anwendung unter  \href{http://www.scratch.mit.edu/}{http://www.scratch.mit.edu/} für die du nichts installieren musst. Klicke auf dafür auf der Webseite auf die Katze mit dem großen grünen Punkt auf dem steht: 'Probier es aus', um den Online-Editor zu starten.


\textbf{WICHTIG:} schalte die Sprache auf Deutsch um. Klicke auf den \glqq Kreis\grqq $\,$ links oben zwischen dem Scratch-Schriftzug und \glqq Datei\grqq $\,$ und wähle  \glqq Deutsch\grqq $\,$ aus. 


\section{Erste Schritte}
Oben in der Mitte des Scratch-Fensters sieht man verschiedenfarbige Kategorien:
\begin{itemize}
\item \textcolor{Bew}{Bewegung}
\item \textcolor{Aus}{Aussehen}
\item \textcolor{Kla}{Klang}
\item \textcolor{Mal}{Malstift}
\item \textcolor{Spe}{Speicher}
\item \textcolor{Ere}{Ereignisse}
\item \textcolor{Ste}{Steuerung}
\item \textcolor{Fue}{F"uhlen}
\item \textcolor{Ope}{Operatoren}
\item \textcolor{Wei}{Weitere Blöcke}
\end{itemize}

In diesem Abschnitt geht es um die Kategorien \textcolor{Spe}{Speicher}, \textcolor{Ste}{Steuerung}, \textcolor{Fue}{F"uhlen} und \textcolor{Ope}{Operatoren}. Nimm dir bitte einen Stift und Papier zur Hand, um dir wichtige Fragen und Antworten zu notieren. Das hilft, dir alles zu merken. Falls du gerade keinen Stift und Papier findest, frag einen der Mentoren.

\subsection{Speicher}
Schaue dir die Kategorie Speicher mal genauer an indem du darauf klickst. Unter den verschiedenen Kategorien erscheinen diesmal keine Blöcke sondern nur zwei Kästchen mit der Aufschrift 'Neue Variable' und 'Neue Liste'.
\begin{itemize}
\item Klicke auf 'Neue Variable' und gebe der neuen Variable einen Namen, z.B. zahl1 und bestätige durch klicken auf $\,$ 'OK'.
\item Welche neuen Blöcke gibt es nun? Überlege dir was diese Blöcke tun.
\item Setzte die Variable auf den Wert 10 und ändere die Variable dann mit einem anderen Block um -5.
\item Was tun die beiden Blöcke \textcolor{Spe}{'zeige Variable'} und \textcolor{Spe}{'verstecke Variable'}?
\item nimm auch einen \textcolor{Aus}{'sage [  ]'} Block aus der Kategorie \textcolor{Aus}{Aussehen} und füge anstelle des Textes (Hello!) die Variable ein (\textcolor{Spe}{ovaler Block} direkt unter 'Neue Variable')
\item Erstelle noch eine Variable, z.B. zahl2, und verwende wieder die Blöcke um die 'zahl2' auf einen Wert zu setzten und dann zu ändern.
\item Wie kann man wieder die Blöcke für 'zahl1' bekommen? Achte wieder auf das kleine schwarze Dreieick in den Blöcken.
\end{itemize}


\subsection{Fühlen}
Die Blöcke in dieser Kategorie haben alle die \textcolor{Fue}{Farbe hellblau}. 
\begin{itemize}
\item Schau dir die Formen der Blöcke an. Welche Formen gibt es?
\item Was ist der unterschied zwischen den runden/spitzen Blöcken und den anderen Blöcken?
\item Was passiert wenn man die runden oder spitzen Blöcke in einen \textcolor{Aus}{'sage [  ]'} Block aus der Kategorie \textcolor{Aus}{Aussehen} einfügt?
\end{itemize}
Die Farbe(n) der kleinen Quadrate der Blöcke \textcolor{Fue}{'wird Farbe [  ] berührt?'} und \textcolor{Fue}{'Farbe [  ] berührt [  ]?'} lassen sich verändern, indem man sie einmal anklickt und dann das Objekt anklickt dessen Farbe man haben will. Das kleine Kästchen sollte dann diese Farbe behalten. Der entsprechende Block muss dazu im Feld rechts der Blöcke sein.
\begin{itemize}
\item Was passiert, wenn man einen \textcolor{Fue}{'wird Farbe [  ]'} Block in den \textcolor{Aus}{'sage [  ]'} Block einfügt und die Farbe auf weiß einstellt? Was wird gesagt?
\item Schaue dir auch nochmal die Blöcke aus den anderen Kategorien \textcolor{Spe}{Speicher}, \textcolor{Bew}{Bewegung}, \textcolor{Aus}{Aussehen} und \textcolor{Ere}{Ereignisse} an und achte dabei auf die unterschiedlichen Formen.
\item welche Formen passen in das weiße Feld des \textcolor{Aus}{'sage [  ] '} Blocks und welche in das des \textcolor{Bew}{'gehe ( )er-Schritt'}?
\end{itemize} 

\subsection{Operatoren}
Auch in der Kategorie \textcolor{Ope}{Operatoren} gibt wieder Blöcke mit verschiedenen Formen.
Die ersten vier Blöcke sind die Mathematischen Grundrechenarten die du sicher aus der Schule kennst. Probiere sie aus indem du sie in einen \textcolor{Aus}{'sage [  ] '} Block ziehst und in die beiden leeren Felder Zahlen schreibst.
\begin{itemize}
\item Was passiert beim nächsten Block: \textcolor{Ope}{'Zufallszahl von ( ) bis ( )'}? Ziehe auch diesen Block in einen \textcolor{Aus}{'sage [  ] '} Block und klicke mehrmals darauf? Was ändert sich in der Sprechblase der Katze? Was ändert sich, wenn du die Zahlen in den weißen runden Felder des Blocks veränderst?
\item Die nächsten drei Blöcke \textcolor{Ope}{( ) < ( )}, \textcolor{Ope}{( ) = ( )} und \textcolor{Ope}{( ) > ( )} vergleichen die Zahlen, die in die beiden Käschen geschrieben werden. 
\newline
\item Überlege dir was die Katze sagt wenn du eingibst:
\begin{itemize}
\item \textcolor{Ope}{'$(5) < (10)$'}
\item \textcolor{Ope}{'$(5) = (10)$'}
\item \textcolor{Ope}{'$(4) > (17)$'}
\item \textcolor{Ope}{'$(7) > (-7)$'}
\item \textcolor{Ope}{'$(3) = (3)$'}
\end{itemize} 
\item Gebe nun die Beispiele von oben ein und überprüfe so ob deine Überlegungen stimmen.
\end{itemize}

\subsection{Steuerung}
Schaue dir nun die Kategorie \textcolor{Ste}{Steuerung} genauer an indem du darauf klickst. Unter den verschiedenen Kategorien erscheinen wieder Blöcke in dieser \textcolor{Ste}{Farbe}.
\begin{itemize}
\item Lies dir die Beschriftungen der Blöcke durch (bis zum Block \textcolor{Ste}{'warte bis < >'}) und überlege was sie tun könnten.
\item Manche Blöcke haben hier wieder eine andere Form. Was ist anders?
\item Was ist der Unterschied der Blöcke \textcolor{Ste}{'wiederhole (10) mal'} und \textcolor{Ste}{'wiederhole fortlaufend'}? Teste deine Vermutung indem du den Block \textcolor{Bew}{'drehe dich um (15) Grad'} aus der Kategorie \textcolor{Bew}{Bewegung} jeweils einmal mit jedem Block verbindest.
\newline
\end{itemize}
Die nächsten beiden Blöcke \textcolor{Ste}{'falls <  > dann'} und \textcolor{Ste}{'falls <  > dann ... sonst'} haben Lücken für spitze Blöcke. Suche dir einen Block aus der in die Lücke passt verbinde dann den \textcolor{Ste}{'falls <  > dann'} oder \textcolor{Ste}{'falls <  > dann ... sonst'} Block mit anderen Blöcken.
\begin{itemize}
\item Was passiert? 
\item Was tun die beiden Blöcke?
\item Frage einen Mentor, wenn du die Blöcke nicht verstehst.
\newline
\end{itemize}

Verbinde nun einige Blöcke aus allen Kategorien miteinander und klicke drauf. Lass die Katze durch drücken der Leertaste z.B. einen Schritt der Größe \textcolor{Spe}{'(zahl1)'} nach vorne gehen und änderen dann  \textcolor{Spe}{'(zahl1)'} um \textcolor{Spe}{'(zahl2)'}. Dann kannst du z.B. ein großes rotes Viereck an den rechten Rand deiner Bühne zeichnen und dann überprüfen ob die Katze Farbe rot berührt und die Katze wieder an den linken Rand deiner Bühne zurücksetzten, falls sie Rot berührt. Schau dazu auch auf die Fragen und deine Antworten des ersten Tutorials. \\
\newline
Falls dein Fenster inzwischen zu voll geworden ist, kannst du einige der Blöcke, die du gerade nicht mehr brauchst wieder zurück in die Mitte zu den anderen Blöcken ziehen. Diese Blöcke verschwinden dann.\\
\newline
Konntest du alle Fragen beantworten? Zeige einem Mentor was du gemacht hast und stelle ihm ruhig die Fragen zu allem.\\
\newline
Nun kannst du die Grundlagen-Blöcke verwenden. Probiere am besten gleich eine Aufgabe dazu aus.
\end{document}
