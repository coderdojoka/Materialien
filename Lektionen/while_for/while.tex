%% Author: Mark weinreuter
\newcommand*{\VorlagenPfad}{../../Vorlagen}%
\documentclass{\VorlagenPfad/coderdojokatext}

% Titel, Kategorie müssen als Kommando für den Header/Footer definiert werden.
\newcommand{\Titel}{while- und for-Schleifen}
\newcommand{\Kategorie}{Basics:\space} 

\begin{document}

{\huge Vorläufige Version, To be improved...}
\begin{center}
	{\huge \Titel}
\end{center}

\section{Wiederholungen} Du hast die Aufgabe ein Programm zu schreiben, das die Zahlen von 1 bis 100 auszugibt. Wie könnte man diese Aufgabe lösen? Einfach hintereinander hinschreiben:

\begin{pythoncode}
print(1)
print(2)
# ... 96 Zeilen mehr
print(99)
print(100)
\end{pythoncode}

Diese Methode ist umständlich und sehr aufwendig. Glücklicherweiße gibt es dafür eine schlaue Lösung.

\section{Die for-Schleife} Die \code{for}-Schleife dient dazu Elemente, z.B. die einer Liste einzeln durchzugehen und aufzulisten.

\begin{pythoncode}
for zahl in range(1, 100):
	print(zahl)
\end{pythoncode}

\code{range(1, 100)} (engl. für Bereich) erzeugt eine Liste mit allen Zahlen zwischen dem Startwert 1 und dem Endwert 100. Diese werden dann nacheinander in die Variable \code{zahl} geschrieben. Daraufhin werden alle eingerückten Zeilen der \code{for}-Schleife ausgeführt.
\\
Der obige Code gibt also alle Zahlen zwischen 1 und 100 aus!

\begin{pythoncode}
for zahl in range(1, 100, 2):
	print(zahl)
\end{pythoncode}
\code{range(1, 100, 2)} erzeugt eine Liste mit allen Zahlen zwischen 1 und 100 in 2-er Schritten. Also \code{1, 3, ..., 97, 99}. Anstelle von \code{2} kann jede beliebige andere Zahl verwendet werden.

\section{Aufgaben}
\begin{itemize}
	\item Schreibe eine \code{for}-Schleife, die alle Zahlen zwischen 40 und 100 in 3-er Schritten ausgibt
	\item Experimentiere mit den Werten von \code{range}. Was passiert, wenn der Startwert größer ist als der Endwert? Was passiert, wenn der 'Hochzählwert' (z.B. die 2 im obigen Beispiel) größer ist als der Endwert. Was passiert, wenn der 'Hochzählwert' negativ ist?
	\item Schreibe eine \code{for}-Schleife, die alle Zahlen zwischen 100 und 1 ausgibt. Also \code{100, 99, ..., 2, 1}.
\end{itemize}

\end{document}