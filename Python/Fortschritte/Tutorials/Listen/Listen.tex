% Author: Norbert Weinreuter		
\newcommand*{\VorlagenPfad}{../../../../Vorlagen}
\documentclass{\VorlagenPfad/coderdojokatext}

\newcommand{\Titel}{Listen}

\begin{document}
	
\begin{center}
	{\huge \Titel}
\end{center}


\section{Was sind Listen?}
In Listen kann man verschiedene Informationen speichern. Zum Beispiel die Wochentage. Der Inhalt einer Liste wird zwischen eckige Klammern gesetzt \code{[….]} und durch Kommata getrennt. 

\begin{pythoncode}
liste1 = ["Montag", "Dienstag", "Mittwoch", "Donnerstag", ]
\end{pythoncode}
Wenn wir jetzt unsere Liste ausdrucken wollen und

\begin{pythoncode}
print (liste1)
\end{pythoncode}

eingeben, erhalten wir diese Bildschirmausgabe:

\begin{pythoncode}
['Montag', 'Dienstag', 'Mittwoch', 'Donnerstag']
\end{pythoncode}

\section{Den Inhalt einer Liste mit der for-Schleife ausgeben}
Wir wollen aber die Tage einzeln angezeigt haben. Dafür benutzen wir eine 
for-Schleife. Die Einträge werden zunächst in eine Variable geschrieben, die wir  z.B. \code{eintrag} nennen und dann wird die \code{for}-Schleife so lange durchlaufen, wie Elemente in der Liste sind und die Einträge untereinander ausgegeben. Die Variable \code{eintrag} nimmt dabei in jedem Listen-Durchlauf die einzelnen Werte der Liste an.
\begin{pythoncode}
for eintrag in liste1:
	print (eintrag)
\end{pythoncode}
erzeugt folgende Ausgabe:
\begin{pseudocode}
Montag
Dienstag
Mittwoch
Donnerstag	
\end{pseudocode}

\section{Anzahl der Elemente mit len() ermitteln}
Um herauszufinden, wieviele Einträge die Liste enthält, können wir die bereits bekannte Funktion \code{len()} einsetzen. Wir schreiben die Anzahl der Elemente in die Variable anzahl und geben diese anschließend aus

\begin{pythoncode}
anzahl = len(liste1)  
print (anzahl) 
\end{pythoncode}
Dies gibt 4 aus, d.h. es sind 4 Elemente in unserer Liste enthalten.
\begin{pythoncode}
\end{pythoncode}
\section{Einen Eintrag (z.B. den dritten) aus einer Liste lesen}	
Wie wir oben mit der \code{len()}-Funktion gesehen haben, enthält unsere Liste 4 Einträge
\\

Jedem Wert in unserer Liste ist ein Index (Plural: Indices) zugeordnet. D.h. jeder Eintrag hat eine Zählnummer, anhand derer wir einzelne Einträge heraussuchen können. Der Index wird in [eckige Klammern gesetzt].	
\\
Wir wollen das dritte Element aus unserer liste1 anzeigen.\\
Wir erinnern uns:
\begin{merkbox}
\textbf{Achtung:} Beim Programmieren fangen wir oft bei 0 mit dem Zählen an.
D.h., dass wir um das dritte Element zu erfahren nicht 3 , sondern 2 in die
eckigen Klammern schreiben müssen.
\end{merkbox}
Die Einträge in unsere Liste haben also folgende Indices.

\begin{table}[width=\textwidth]
	\begin{tabular}{lllll}
		\mintinline{python}{liste1 =} &
		\mintinline{python}{["Montag",}  	& \mintinline{python}{"Dienstag",}         &		\mintinline{python}{"Mittwoch",} & \mintinline{python}{"Donnerstag"]}               \\
		
		Index:  & \multicolumn{1}{c}{0} & \multicolumn{1}{c}{1} & \multicolumn{1}{c}{2} & \multicolumn{1}{c}{3} \\
	\end{tabular}
\end{table}

Geben wir das dritte Element nun aus:
\begin{pythoncode}
print (liste1[2])      # das dritte Element unserer Liste anzeigen	
\end{pythoncode}
erhalten wir:
\begin{pseudocode}
Mittwoch
\end{pseudocode}

\section{Listen von hinten lesen}
Man kann Listen auch vom rechts nach links lesen. 

Hierfür werden \textit{negative Indices} beginnend beim \underline{letzten} Element mit \code{[-1]} benutzt:

\begin{table}[width=\textwidth]
	\begin{tabular}{lllll}
		\mintinline{python}{liste1 =} &	\mintinline{python}{["Montag",} & \mintinline{python}{"Dienstag",} & \mintinline{python}{"Mittwoch",} & \mintinline{python}{"Donnerstag"]}               \\
		
		Index:  & \multicolumn{1}{c}{-4} & \multicolumn{1}{c}{-3} & \multicolumn{1}{c}{-2} & \multicolumn{1}{c}{-1} \\
		\end{tabular}
\end{table}

ein anderes Beispiel:

\begin{table}[width=\textwidth]
	\begin{tabular}{lllll}
		\mintinline{python}{liste =} &
		\mintinline{python}{["Hallo",}  & \mintinline{python}{"Test",} & \mintinline{python}{"Welt"]}  \\
		
		Index:  & \multicolumn{1}{c}{0} & \multicolumn{1}{c}{1} & \multicolumn{1}{c}{2} \\          
		neg. Index:  & \multicolumn{1}{c}{-3} & \multicolumn{1}{c}{-2} & \multicolumn{1}{c}{-1} \\
	\end{tabular}
\end{table}

Verwenden wir nun unser Wissen können wir das erste und letzte Element ganz einfach ausgeben:
\begin{pythoncode}
# Ein Eintrag kann so gelesen werden:
ersterEintrag = liste[0] # = "Hallo". Achtung wir beginnen bei 0 !

# Negative Indices beginnen am Ende zu zählen
letzterEintrag = liste[-1] # = "Welt". Wir beginnen bei -1 !

print(ersterEintrag, letzterEintrag) # => "Hallo Welt"
\end{pythoncode}

\section{Einen Eintrag ans Ende der Liste einfügen}
Wollen wir jetzt den Freitag zur \code{liste1} hinzufügen, kann das mit der \code{.append(...)}-Funktion geschehen.
\\
Wir sagen, wo wir etwas hinzufügen (\code{append} engl. für 'ans Ende anfügen') wollen: an das Ende der \code{liste1}.
Was wir hinzufügen wollen (Freitag), schreiben wir in runde Klammern:
\begin{pythoncode}
liste1.append("Freitag") 
\end{pythoncode}
Geben wir nun die Liste wie oben beschrieben mithilfe ein \code{for}-Schleife aus, so erhalten wir:
\begin{pseudocode}
Montag
Dienstag
Mittwoch
Donnerstag	
Freitag
\end{pseudocode}
\end{document}