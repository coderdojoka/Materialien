% Die folgenden zwei Befehle binden die CoderDojo-Styles ein.
		
\newcommand*{\VorlagenPfad}{../../../../Vorlagen}	
\documentclass{\VorlagenPfad/coderdojokatext}

% Titel als Kommando für den Header/Footer definieren
\renewcommand{\Titel}{py2cd - Zeichnen mit Python Teil III}
\usepackage{hyperref}

\begin{document}

% Titel anzeigen und um den Header/Footer zu generieren
\begin{center}
	{\huge \Titel}
\end{center}

\section{Vorbereitung}
Um dieses Tutorial erfolgreich abzuschließen, muss zuerst pygame und py2cd installiert sein. Eine Installationsanleitung ist unter \url{https://github.com/coderdojoka/py2cd/} zu finden.
\\
Desweiteren sollte Teil I und Teil II dieser Tutorial Serie verstanden sein.
\section{Bewegung}

\begin{merkbox}
\textbf{Erinnerung:} Der Koordinatenursprung \code{(0|0)} liegt in der linken oberen Ecke.
Eine Änderung der x-Koordinate, die größer 0 ist, ist eine Bewegung nach rechts. Eine negative Änderung bedeutet eine Bewegung nach links.	
\\
Eine Änderung der y-Koordinate, die größer 0 ist, ist eine Bewegung nach \textbf{unten!}. Dies ist zunächst verwirrend, ist allerdings bei vielen Computerprogrammen so. Eine negative Änderung bedeutet demzufolge eine Bewegung nach oben!	

\begin{pythoncode}
box = Rechteck(10, 10, 50, 50, ROT)
# Ändert die x- und y-Koordinate um 5 Pixel
box.aendere_position(5, 5)
\end{pythoncode}

Dies ist also eine Bewegung um jeweils 5 Pixel nach rechts unten!
\end{merkbox}

\subsection{Objekte positionieren}
Es gibt verschiedene Möglichkeiten Objekte, wie Kreise, Rechtecke, Bilder, etc.. zu platzieren:

\subsubsection{Mit links, rechts, oben, unten:}
Bei der Definition wird meistens die x-, y-Koordinate mit angegeben. Diese kann aber auch noch leicht nachträglich geändert werden, indem der Abstand zum linken, rechten, oberen oder unteren Rand angegeben wird.

\inputminted[firstline=10,lastline=24]{python}{../../../Beispiele/py2cd/objekte_positionieren.py}

\subsubsection{Die Objekte-Mitte festlegen:}
Man kann auch die Objektmitte direkt setzen:

\inputminted[firstline=26,lastline=27]{python}{../../../Beispiele/py2cd/objekte_positionieren.py}

\subsection{Die x-,y-Koordinaten ändern:}

\subsubsection{Relative Änderung}
Die Position um einen Wert ändern. \code{x\_neu = x\_alt + wert}.

\begin{pythoncode}
box = Rechteck(10, 10, 50, 50, ROT)
# Ändert die x- und y-Koordinate um 5 Pixel
box.aendere_position(5, 5)
\end{pythoncode}

\subsubsection{Absolute Position setzen}
Die Position auf einen Wert setzen. \code{x\_neu = x\_alt}.

\begin{pythoncode}
box = Rechteck(10, 10, 50, 50, ROT)
# Setzt die x- und y-Koordinate auf die gegebenen Werte
box.setze_position(50, 50)
\end{pythoncode}

\paragraph{Hinweis:} Diese Bewegungsmöglichkeiten sind ähnliche wie die in Scratch. \code{aendere\_position} verhält sich genau wie \code{'ändere x um ...'} und \code{'ändere y um ...'}.
\\
\code{setze\_position} verhält sich genau wie \code{'setze x auf ...'} und \code{'setze y auf ...'}.

\subsection{Objekte zentrieren:}
Objekte können horizontal, vertikal oder in beide Richtungen zentriert werden.

\inputminted[firstline=29,lastline=38]{python}{../../../Beispiele/py2cd/objekte_positionieren.py}

\subsection{Geschwindigkeit festlegen und bewegen}
Man kann Objekte auch mithilfe von \code{bewege()} bewegen. Dabei wird das Objekt um die mit \code{setze\_geschwindigkeit(6,6)} definierte Distanz (hier: 6 Pixel nach rechts und 6 Pixel nach unten).

\begin{pythoncode}
box = Rechteck(10, 10, 50, 50, ROT)
# Die x- und y-Geschwindigkeit setzen
box.setze_geschwindigkeit(6,6)
# Bewegt die Box bei jedem Aufruf von bewege() um 6 Pixel
box.bewege()
\end{pythoncode}



\subsection{Position abrufen}
\subsubsection{x- und y-Koordinaten}
Man kann die x- und y-Koordinaten eines Objektes ganz einfach abfragen
\begin{pythoncode}
# Box an der Stelle 10x10
box = Rechteck(10, 10, 50, 50, ROT)
# x- und y-Koordinate abfragen und ausgeben
print(box.x, box.y)
\end{pythoncode}

\subsubsection{Abstand zum Rand}
Genau wie man den Abstand zum linken, rechten, oberen  oder unteren Rand setzen kann, kann man ihn auch abfragen. Das Gleiche gilt auch für die Objektmitte.

\inputminted[firstline=42,lastline=60]{python}{../../../Beispiele/py2cd/objekte_positionieren.py}

\subsubsection{Größe des umgebenden Rechtecks}
Wie zuvor beschrieben, ist jedes Objekt von einem unsichtbaren Rechteck eingegrenzt. Man kann auch dessen Breite und Höhe abfragen.

\inputminted[firstline=62,lastline=68]{python}{../../../Beispiele/py2cd/objekte_positionieren.py}

\section{Aufgaben}
\begin{enumerate}
\item Bewege ein Objekt mit den verschiedenen Funktionen über die Spielfläche. 
\item Frage die Position von Objekten und derren Abstand zum Rand ab. Ändert sich die Postion wie erwartet, wenn du sie bewegst? Z.B. sollte sich die x-Koordinate um 20 ändern, wenn \code{aendere\_position(20, 0)} aufrufst.
\end{enumerate}



\end{document}