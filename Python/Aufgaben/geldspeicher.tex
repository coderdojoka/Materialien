% Vorlage verwenden. ACHTUNG der Pfad ist relativ und geht davon aus, 
% diese Datei zwei Ordnerebenen höher liegt als Vorlagen
\newcommand*{\VorlagenPfad}{../../Vorlagen}%
\documentclass{\VorlagenPfad/coderdojokatext}

% Titel als Kommando für den Header/Footer definieren
\newcommand{\Titel}{Der Geldspeicher}

\begin{document}
	\setcounter{chapter}{1}
	
	\begin{center}
		{\huge \Titel}
	\end{center}
	
	\section{Die Aufgabenstellung} Eines Tages stellt Onkel Dagobert fest, dass er sein Notizzettel verloren hat. Dort hat er das Passwort für seinen Geldspeicher notiert. Nun hat er Angst davor, dass die bösen Panzerknacker seine Reichtümer verschleppen und will deshalb sein Geld in Sicherheit bringen.
	
	Onkel Dagobert ist ganz verzweifelt und hofft auf deine Hilfe. Versuche ein Programm zu schreiben, welches ihm beim Öffnen des Geldspeichers hilft.
	
	\subsection{Vorgabe} Daniel Düsentrieb hat ein Hilfsprogramm entwickelt, mit dem du ein dreistelliges Passwort (nur Ziffern) im Sicherheitsfeld eingeben kannst. Hierzu musst du das Modul \emph{Geldspeicher} importieren und die Methode \emph{rate\_passwort()} mit deinem geratenen Passwort aufrufen.
	
	\subsection{Vorrausetzungen} Um diese Aufgabe erfolgreich lösen zu können solltest du dich mit Schleifen auskennen und die Aufgaben zu \code{for}-Schleifen und geschachtelten \code{for}-Schleifen gemacht haben.
	
	\section{Umsetzung}
	Überlege dir eine Technik, mit der du möglichst geschickt die richtige Kombination eingeben kannst.\\
	Um die Aufgabe zu lösen benötigst du die Dateien \code{geldspeicher.py} und \\
	\code{geldspeicher\_test.py}. Hier der Inhalt von\code{geldspeicher\_test.py}. Schreibe in diese Datei deine Lösung.	
	\inputminted{python}{code/geldspeicher_test.py}
	
	Hinweis: Du kannst beliebig viele Falscheingaben ausprobieren.\\
	\\
	Die Methode \emph{rate\_passwort()} gibt wahr oder falsch zurück. Außerdem erscheint die Ausgabe \emph{'Geldspeicher geöffent'}, wenn du die richtige Kombination gefunden hast.
	\\
	Onkel Dagobert wird sich in diesem Fall bestimmt erkenntlich zeigen.
	
\end{document}