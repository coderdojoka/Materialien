%% Author: Mark weinreuter
\newcommand*{\VorlagenPfad}{../Vorlagen}%
\documentclass{\VorlagenPfad/coderdojokatext}

% Titel, Kategorie müssen als Kommando für den Header/Footer definiert werden.
\newcommand{\Titel}{Python - Befehlsreferenz}

\begin{document}
\setcounter{chapter}{1}

\begin{center}
	{\huge \Titel}
\end{center}

\section{Allgemein}
\subsection{Kommentare}
\begin{pythoncode}
# Ein Kommentar, der bis ans Zeilenende geht
"""
Ein Kommentar, der über
mehrere Zeilen geht.
"""
\end{pythoncode}
\subsection{Variablen}
\begin{pythoncode}
# Eine neue Variable Zahl mit dem Namen 'meineZahl' und Wert 42
meineZahl = 42

# Eine andere Variable mit einem Text-Wert, Texte müssen in ".." stehen.
meinText = "Hallo Welt"
\end{pythoncode}

\subsection{Ausgabe}

\begin{pythoncode}
# Gibt einen Text aus
print("Hallo Welt")

\end{pythoncode}
\begin{pythoncode}
# Gibt den Wert einer Variablen aus
name = "Mark"
print("Hallo")
print(name)

# 'print' akzeptiert mehrere Werte gleichzeitig
print("Hallo", name)
\end{pythoncode}

\subsection{Operationen}
\begin{pythoncode}
wert  = 10 # Mit Zahlen kann ganz normal rechnen
wert = 10 + 20  # = 30
wert = 10 - 20  # = -10
wert = 10 / 2   # = 5
wert = 10 * 2   # = 20
\end{pythoncode}


\begin{pythoncode}
# Texte kann man '+' aneinander anfügen
begruessung  = "Hallo "
text = begruessung + "Mark" # "Hallo Mark"
# Texte kann man auch multiplizieren/wiederholen
text = "Ha" * 3  # HaHaHa
\end{pythoncode}

\section{Schleifen}

\subsection{Bedingung}
\begin{pythoncode}
# Eine Bedingung kann entweder Wahr (True) oder Falsch (False) sein
bedingung = 10 > 20 # False
print("Die Bedingung ist:", bedingung)
\end{pythoncode}

\subsection{if-Abfragen}
\begin{pythoncode}
wert = 10 # Eine Variable mit einem beliebigen Wert (hier 10)

# eine if-Abfrage überprüft die Bedingung ob diese Wahr oder falsch ist
if wert > 20: 
	# wird ausgeführt wenn die Bedingung wahr ist (ACHTUNG: Einrückung)
	print("Der Wert ist größer als 20")

# Hier gehts weiter (ACHUTUNG: Nicht eingerückt)
\end{pythoncode}

\begin{pythoncode}
# mit einer if-else Abfrage kann man auch auf eine nicht erfüllte
# Bedingung mit dem 'else'-Zweig reagieren
if wert > 20: 
	# wird ausgeführt wenn die Bedingung erfüllt ist
	print("Der Wert ist größer als 20") 
else:
	# wird ausgeführt wenn die Bedingung nicht erfüllt ist
	print("Wert ist kleiner oder gleich 20.")

# Hier gehts weiter (nicht eingerückt)
\end{pythoncode}

\subsection{while-Schleifen}
\begin{pythoncode}
# Führt solange den eingerückten Code aus, wie die Bedingung erfüllt ist
zahl = 1
while zahl < 10: 
	# dieser eingerückte Code wird wiederholt (ACHUTUNG: Einrückung)
	print("Zahl:", zahl) # Wert ausgeben
	zahl = zahl +1 # Zähler erhöhen

# Hier gehts weiter (ACHTUNG: nicht eingerückt)
\end{pythoncode}

\section{Eingabe durch den Benutzer}
\subsection{input - Text einlesen}
\begin{pythoncode}
# mit input(..) kann man den Benutzer aufforden einen Text einzugeben
eingabeText = input("Bitte Text eingeben: "); 
print(eingabeText); # gibt den eingelesenen Text aus
\end{pythoncode}

\subsection{input - Eine Zahl einlesen}
\begin{pythoncode}
# mit input(..) kann man den Benutzer aufforden einen Text einzugeben
eingabeText = input("Bitte eine Zahl eingeben: "); 

# Diesen Text wollen wir mittels int(..) in eine Zahl konvertieren:
eingabeZahl = int(eingabeText) 
# ACHTUNG: wird keine Zahl eingeben, so stürzt das Programm ab
print(eingabeZahl); # gibt die eingelesene Zahl aus
\end{pythoncode}


\section{Zufallszahlen}

\begin{pythoncode}
# das random Modul muss (einmal) importiert werden um die Funktion
# random.randint() verwenden zu können
import random 

# generiert eine Zufallszahl zwischen 1 und 100
zufallsZahl = random.randint(1,100)
\end{pythoncode}


\end{document}