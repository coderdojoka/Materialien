%% Author: Mark weinreuter
\newcommand*{\VorlagenPfad}{../../Vorlagen}%
\documentclass{\VorlagenPfad/coderdojokatext}

% Titel, Kategorie müssen als Kommando für den Header/Footer definiert werden.
\newcommand{\Titel}{Python - Befehlsübersicht}


% hack to force a chapter to be on the same page. This has some issues with the header
\usepackage{etoolbox}
\makeatletter
\patchcmd{\chapter}{\if@openright\cleardoublepage\else\clearpage\fi}{}{}{}
\makeatother

\begin{document}
\setcounter{chapter}{0}

\begin{center}
	{\huge \Titel}
\end{center}

\chapter{Grundlagen}
\section{Kommentare}
\begin{pythoncode}
# Ein Kommentar, der bis ans Zeilenende geht
"""
Ein Kommentar, der über
mehrere Zeilen geht.
"""
\end{pythoncode}
\section{Variablen}
\begin{pythoncode}
# Eine neue Variable Zahl mit dem Namen 'meineZahl' und Wert 42
meineZahl = 42

# Eine andere Variable mit einem Text-Wert, Texte müssen in ".." stehen.
meinText = "Hallo Welt"
\end{pythoncode}

\section{Ausgabe}

\begin{pythoncode}
# Gibt einen Text aus
print("Hallo Welt")

\end{pythoncode}
\begin{pythoncode}
# Gibt den Wert einer Variablen aus
name = "Mark"
print("Hallo")
print(name)

# 'print' akzeptiert mehrere Werte gleichzeitig
print("Hallo", name)
\end{pythoncode}

\section{Operationen}
\subsection{Umgang mit Zahlen}
\begin{pythoncode}
# Mit Zahlen kann ganz normal rechnen
wert = 10 
wert = 10 + 20	# = 30
wert = 10 - 20	# = -10
wert = 10 / 2	# = 5
wert = 10 * 2	# = 20
wert = 10 % 3	# = 1, Da 10 / 3 = 3 Rest 1	
\end{pythoncode}

\subsection{Umgang mit Text}
\begin{pythoncode}
# Texte kann man '+' aneinander anfügen
begruessung  = "Hallo "
text = begruessung + "Mark" # "Hallo Mark"
# Texte kann man auch multiplizieren/wiederholen
text = "Ha" * 3  # HaHaHa
\end{pythoncode}

\section{if-Abfragen}

\subsection{Bedingung}
\begin{pythoncode}
# Eine Bedingung kann entweder Wahr (True) oder Falsch (False) sein
bedingung = 10 > 20 # False
print("Die Bedingung ist:", bedingung)
\end{pythoncode}
\subsection{if}
\begin{pythoncode}
wert = 10 # Eine Variable mit einem beliebigen Wert (hier 10)

# eine if-Abfrage überprüft die Bedingung ob diese Wahr oder falsch ist
if wert > 20: 
	# wird ausgeführt wenn die Bedingung wahr ist (ACHTUNG: Einrückung)
	print("Der Wert ist größer als 20")

# Hier gehts weiter (ACHUTUNG: Nicht eingerückt)
\end{pythoncode}
\subsection{if-else}
\begin{pythoncode}
# mit einer if-else Abfrage kann man auch auf eine nicht erfüllte
# Bedingung mit dem 'else'-Zweig reagieren
if wert > 20: 
	# wird ausgeführt wenn die Bedingung erfüllt ist
	print("Der Wert ist größer als 20") 
else:
	# wird ausgeführt wenn die Bedingung nicht erfüllt ist
	print("Wert ist kleiner oder gleich 20.")

# Hier gehts weiter (nicht eingerückt)
\end{pythoncode}



\section{Schleifen}

\subsection{while-Schleife}
\begin{pythoncode}
# Führt solange den eingerückten Code aus, wie die Bedingung erfüllt ist
zahl = 1
while zahl < 10: 
	# dieser eingerückte Code wird wiederholt (ACHUTUNG: Einrückung)
	print("Zahl:", zahl) # Wert ausgeben
	zahl = zahl +1 # Zähler erhöhen

# Hier gehts weiter (ACHTUNG: nicht eingerückt)
\end{pythoncode}

\subsection{for-Schleife}
\begin{pythoncode}
# Führt solange den eingerückten Code aus, wie die Bedingung erfüllt ist

for zahl in range(1, 10): # In nimmt Werte 1-9 an
	# dieser eingerückte Code wird wiederholt (ACHUTUNG: Einrückung)
	print("Zahl:", zahl) # Wert ausgeben

# Hier gehts weiter (ACHTUNG: nicht eingerückt)
\end{pythoncode}

\section{Eingabe durch den Benutzer}
\subsection{input - Text einlesen}
\begin{pythoncode}
# mit input(..) kann man den Benutzer aufforden einen Text einzugeben
eingabeText = input("Bitte Text eingeben: "); 
print(eingabeText); # gibt den eingelesenen Text aus
\end{pythoncode}

\subsection{input - Eine Zahl einlesen}
\begin{pythoncode}
# mit input(..) kann man den Benutzer aufforden einen Text einzugeben
eingabeText = input("Bitte eine Zahl eingeben: "); 

# Diesen Text wollen wir mittels int(..) in eine Zahl konvertieren:
eingabeZahl = int(eingabeText) 
# ACHTUNG: wird keine Zahl eingeben, so stürzt das Programm ab
print(eingabeZahl); # gibt die eingelesene Zahl aus
\end{pythoncode}


\section{Zufallszahlen}

\begin{pythoncode}
# das random Modul muss (einmal) importiert werden um die Funktion
# random.randint() verwenden zu können
import random 

# generiert eine Zufallszahl zwischen 1 und 100
zufallsZahl = random.randint(1,100)
\end{pythoncode}






\chapter{Fortgeschritte Grundlagen}
% next chapter :D

\section{Geschachtelte For-Schleifen}
\begin{pythoncode}
# geschachtelte for-Schleife, die die innere Schleife 9 mal ausführt	
for zahlAussen in range(1, 10): 	# nimmt Werte 1-9 an
	for zahlInnen in range(1, 10):	# nimmt Werte 1-9 an
		# beide Wert ausgeben
		print("Äußere Zahl:", zahlAussen, "Innere Zahl:", zahlInnen) 
\end{pythoncode}

\section{Funktionen}

\subsection{Eine einfache Funktion}
\begin{pythoncode}
# Funktion definieren
def sageHallo():
	print("Hallo Welt")
	
# Funktion aufrufen
sageHallo()	
\end{pythoncode}

\subsection{Funktion mit Übergabeparametern}
\begin{pythoncode}
# Funktion definieren
def sageHallo(name, alter):
	print("Hallo", name)
	print("Du bist", alter, "Jahre alt")

# Funktion aufrufen
sageHallo("Mark", 22)	
\end{pythoncode}

\subsection{Funktion mit Rückgabewerten}
\begin{pythoncode}
# Funktion definieren
def addiere(zahl1, zahl2):
	summe = zahl1 + zahl2
	return summe

# Funktion aufrufen
ergebnis = addiere(12, 22)
print(ergebnis)
\end{pythoncode}


\section{Listen}

\subsection{Eine Liste erstellen}
\begin{pythoncode}
# eine Liste erstellen
liste = [1, 2, 3, 4, 5]
\end{pythoncode}


\subsection{Eine Liste ausgeben}
\begin{pythoncode}
	
# die Elemente einer Liste könne so ausgeben werden
for element in liste:
	print(element)
	
# Hilfsfunktion für das Ausgeben einer Liste
def liste_ausgeben(liste):	
	print("Liste mite", len(liste), "Einträgen")
	# die Elemente ausgeben
	for element in liste:
	print(element)
	
# Diese ruft man dann so auf
liste_ausgeben(liste)
\end{pythoncode}

\subsection{Einen Eintrag aus der Liste lesen}
\begin{pythoncode}
liste = ["hallo", "test", "welt"]	
# Ein Eintrag kann so gelesen werden:
ersterEintrag = liste[0] # = "hallo". Achtung wir beginnen bei 0 !

# Negative Indices beginnen am Ende zu zählen
letzterEintrag = liste[-1] # = "welt". Wir beginnen bei -1 !

print(ersterEintrag, letzterEintrag) # => "Hallo Welt"

\end{pythoncode}

\subsection{Die Liste erweitern}
\subsubsection{Einen Eintrag hinzufügen}
\begin{pythoncode}
# einen Eintrag anfügen
liste.append(6)
liste_ausgeben(liste)
\end{pythoncode}

\subsubsection{Zwei Listen kombinieren}
\begin{pythoncode}
# Liste mit einer anderen Liste kombinieren
liste = liste + [7, 8, 9]
liste_ausgeben(liste)
\end{pythoncode}




\end{document}