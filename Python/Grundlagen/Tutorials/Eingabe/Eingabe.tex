% Author: Mark Weinreuter		
\newcommand*{\VorlagenPfad}{../../../../Vorlagen}
\documentclass{\VorlagenPfad/coderdojokatext}

\newcommand{\Titel}{Eingaben einlesen}

\begin{document}


\begin{center}
	{\huge \Titel}
\end{center}

\section{Eingaben}
Ein Program, in den der Benutzer nichts eingeben kann ist meinstens langweilig.
Deswegen ist es praktisch, dass wir relativ einfach die Eingaben einlesen können.\\

Dies funktioniert mithilfe von \code{input()}.

\inputminted[firstline=1, lastline=2]{python}{../../../Beispiele/eingabe.py}
Was in den Klammern bei \code{input(...)} steht wird zuerst auf der Konsole ausgeben, dann wird auf einen Eingabe des Benutzers gewartet.\\
\textbf{Wichtig:} das Programm wartet solange bis der Benutzer etwas auf der Konsole eingetippt und mit 'Enter' bestätigt hat!

\inputminted[firstline=2, lastline=2]{python}{../../../Beispiele/eingabe.py}

Wie in dem Tutorial zu Variablen schon erklärt wurde, ist \code{name} eine Variable. Dieser Variable wird der Wert zugewiesen, 
den der Benutzer eintippt und mittels \code{input()} gelesen wird.

\section{Zahlen einlesen}
Mittles \code{input()} kann nur Text eingelesen werden! Auch wenn der Benutzer \code{12} eintippt, so wird dies als Text \code{"12"} interpretiert.
Wir können mit Text allerdings nicht rechnen. Aus diesem Grund müssen wir den eingelesenen Text mithilfe von \code{int(...)} in eine ganze Zahl konvertieren.

\inputminted[firstline=4, lastline=5]{python}{../../../Beispiele/eingabe.py}
Hier wird nun zuerst das Alter als Text eingelesen und dann in eine Zahl umgewandelt. Die Variable \code{alter} enthält zunächst also einen Text und danach die Zahl, die dieser Text darstellt.

\section{Aufgaben}
\begin{itemize}
	\item Lies deinen Namen von der Konsole ein und gib diesen aus.
	\item Was passiert, wenn du einen Text in eine Zahl umwandelst und der Text ist keine Zahl?
	\item Lies dein Alter ein und berechne wie viele Monate du ungefähr alt bist
\end{itemize}

\end{document}