% Author: Mark Weinreuter		
\newcommand*{\VorlagenPfad}{../../../../Vorlagen}
\documentclass{\VorlagenPfad/coderdojokatext}

\renewcommand{\Titel}{Text}

\begin{document}


\begin{center}
	{\huge \Titel}
\end{center}

\section{Warum Text}
An vielen Stellen beim Programmieren hat man mit Text zu tun. Man muss Texte lesen, auf bestimmte Buchstaben untersuchen, etc...
\inputminted[firstline=1, lastline=3]{python}{../../../Beispiele/text.py}

Textstücke erkennt man daran, dass sie in einfachen \code{'...'} oder doppelten \code{"..."} Anführungszeichen  stehen.
\section{Operatoren}
Man auch einzelne Textstücke kombinieren.
\inputminted[firstline=5, lastline=7]{python}{../../../Beispiele/text.py}
Dies kann man auch mit \code{print()} direkt ausgeben, da mehrere Parameter übgergeben werden können.

\inputminted[firstline=9, lastline=10]{python}{../../../Beispiele/text.py}

Analog zum Plus-Operator gibt es auch den Mal-Operator
\inputminted[firstline=12, lastline=13]{python}{../../../Beispiele/text.py}

\section{Länge eines Textes}
Manchmal ist es praktisch zu wissen aus wie vielen Buchstaben ein Text besteht. Wir nennen dies auch die Länge eines Textes.
Um die Anzahl der Buchstaben, eines Textes zu ermitteln, benutzt man den Befehl \code{len(...)} ('length' ist Englisch für Länge) wie folgt:
\inputminted[firstline=15, lastline=16]{python}{../../../Beispiele/text.py}

\section{Texte als Buchstabenketten}
Im Rechner intern, werden Texte als ein Kette von einzelnen Buchstaben gespeichert. Dies können wir uns zu Nutzen machen.

\inputminted[firstline=18, lastline=21]{python}{../../../Beispiele/text.py}

Mit den eckigen Klammern am Ende der Text-Variablen \code{meinText[2]} deuten wir an, 
dass wir den Buchstaben wissen wollen, der an der in Klammern geschrieben Stelle steht.

\begin{merkbox}
	\textbf{Achtung:} Beim Programmieren fangen wir oft bei \code{0} mit dem Zählen an.\\
	D.h., dass wir um den dritten Buchstaben zu erfahren nicht \code{3}, sondern \code{2} in die eckigen Klammern schreiben müssen.
\end{merkbox}

\section{Texte sind unveränderlich}
Eine interessante Eigenschaft von Texten ist, dass diese nicht verändert werden können. 
Wenn wir versuchen, z.B. den vierten Buchstaben zu überschreiben, erhalten wir einen Fehler:

\inputminted[firstline=22, lastline=23]{python}{../../../Beispiele/text.py}

\section{Aufgaben}
\begin{itemize}
	\item Erstelle eine Text-Variable und las dir den Wert ausgeben.
	\item Spiele mit den Operatoren herum. Wie kannst du 100-mal \code{"HaHa"} ausgeben?
	\item Lass dir den zweiten und vorletzen (Hinweis: berechne die Länge des Textes) Buchstaben von \code{"Hallo Welt"}  ausgeben
	\item Probiere wie oben beschrieben den Text zu ändern. Bekommst du auch einen Fehler?
\end{itemize}

\end{document}