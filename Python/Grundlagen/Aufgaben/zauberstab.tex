% Vorlage verwenden. ACHTUNG der Pfad ist relativ und geht davon aus, 
% diese Datei zwei Ordnerebenen höher liegt als Vorlagen
\newcommand*{\VorlagenPfad}{../../Vorlagen}%
\documentclass{\VorlagenPfad/coderdojokatext}

% Titel als Kommando für den Header/Footer definieren
\newcommand{\Titel}{Der Zauberstab}

\begin{document}
	
	
	\begin{center}
		{\huge \Titel}
	\end{center}
	
	\section{Die Aufgabenstellung} Du hast einen Brief bekommen mit einer Einladung: Du darfst in Hogwarts zaubern lernen! Doch erstmals musst du einkaufen gehen, denn was ist ein Magier ohne Eule, Besen und Zauberstab?
	\\ \\
	Im Zauberstabladen bekommst du vom Zauberstabhersteller Ollivander alle seine Zauberstäbe nacheinander vorgelegt. Du willst natürlich der größte Magier aller Zeiten werden und musst deshalb den stärksten aller Stäbe auswählen!
	\\ \\
	Ollivander zeigt dir nacheinander die Zauberstäbe und sagt zu jedem Zauberstab aus welchem Material dieser gemacht ist und wie stark er ist.
	Du hörst zu, bis Olivander das Wort "`fertig"' sagt und antwortest dann mit dem Material des stärksten Zauberstabes, der dir gezeigt wurde.
	\\ \\
	Schreibe ein Programm, in das du die von Ollivander genannten Materialen und Stärken einträgst. Sobald "`fertig"' eingeben wird, sollst du das stärkste eingegeben Material ausgeben.
	\\
	\\
	Hinweis: Die Stärke aller Zauberstäbe ist größer als 0!
	
	\subsection{Vorrausetzungen} Um diese Aufgabe erfolgreich lösen zu können solltest du dich mit \code{if}-Bedingungen, Variablentypen und \code{while}-Schleifen auskennen.
	\subsection{Vorüberlegungen}
	Welche Schritte sind nötig um das Problem zu lösen?
	\begin{itemize}
		\item Wie können Eingaben eingelesen werden und welche werden benötigt?
		\item Wie lange werden Eingaben gelesen und wann wird abgebrochen?
		\item Was soll ausgegeben werden?
	\end{itemize}

\subsection{Ablauf-Beispiele}
Hier findest du 2 Beispiele wie das Programm ablaufen könnte. 
\subsubsection{Beispiel 1}
\begin{pseudocode*}{linenos=false}
# Es wird eingegeben:
Zeder
3
Bambus
2 
Buche 
6 
Plastik 
1 
Phönixfeder 
10 
fertig

# Du gibst auf der Konsole aus: 
Phönixfeder

\end{pseudocode*}

\subsubsection{Beispiel 2}

\begin{pseudocode*}{linenos=false}
# Es wird eingegeben:
Diamant
100
Pappe
2
fertig

# Du gibst auf der Konsole aus:
Diamant
\end{pseudocode*}

Schaffst du es den stärksten Zauberstab für dich zu finden?
	
\end{document}