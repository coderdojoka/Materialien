% Vorlage verwenden. ACHTUNG der Pfad ist relativ und geht davon aus, 
% diese Datei zwei Ordnerebenen höher liegt als Vorlagen
\newcommand*{\VorlagenPfad}{../../Vorlagen}%
\documentclass{\VorlagenPfad/coderdojokatext}
\usepackage{graphicx}

% Titel als Kommando für den Header/Footer definieren
\newcommand{\Titel}{3....2....1...}

\begin{document}
	\setcounter{chapter}{1}
	
	\begin{center}
		{\huge \Titel}
	\end{center}
	
	\section{Die Aufgabenstellung} Daniel Düsentrieb hat eine Rakete gebaut, mit der er zum Mond fliegen will um dort etwas Urlaub zu machen. Doch er hat das wichtigste vergessen: den Countdown!
	\\
	\begin{figure}[h]
	\includegraphics[scale=0.8]{rakete}
	\centering
	\end{figure}
	\\ \\
	Hilf dem zerstreuten Daniel, damit auch er seinen Urlaub auf dem Mond genießen kann. Daniel sagt dir, von welcher Zahl du anfangen sollst runter zu zählen. Du sollst für ihn zählen und am Ende laut "`START!"' rufen.
	
	\subsection{Voraussetzungen} Um diese Aufgabe erfolgreich lösen zu können solltest du dich mit \code{for}- oder \code{while}-Schleifen auskennen.
	
	\subsection{Beispiele}
	\subsubsection{Beispiel 1}
	Daniel tippt ein: \\ \\
	3
	\\ \\
	Du gibst auf der Konsole aus: \\ \\
	3... \\
	2... \\
	1... \\
	START!
	
	\subsubsection{Beispiel 2}
	Daniel tippt ein: \\ \\
	10
	\\ \\
	Du gibst auf der Konsole aus: \\ \\
	10... \\
	9... \\
	8... \\
	7... \\
	6... \\
	5... \\
	4... \\
	3... \\
	2... \\
	1... \\
	START!
	
\end{document}