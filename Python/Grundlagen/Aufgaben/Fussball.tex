% Vorlage verwenden. ACHTUNG der Pfad ist relativ und geht davon aus, 
% diese Datei zwei Ordnerebenen höher liegt als Vorlagen
\newcommand*{\VorlagenPfad}{../../Vorlagen}%
\documentclass{\VorlagenPfad/coderdojokatext}
\usepackage{graphicx}

% Titel als Kommando für den Header/Footer definieren
\newcommand{\Titel}{Fußball}

\begin{document}
	\setcounter{chapter}{1}
	
	\begin{center}
		{\huge \Titel}
	\end{center}
	
	\section{Die Aufgabenstellung} Deine Freunde und du haben nie Zeit Fußball zu gucken, weil ihr immer Hausaufgaben machen und für Klassenarbeiten lernen müsst. Aber du kannst aus deinem Zimmer den Fernseher hören! Du bekommst also immer mit, wann eine Mannschaft ein Tor schießt.
	\\
	\begin{figure}[h]
	\includegraphics[scale=0.4]{Fussball_Bild}
	\centering
	\end{figure}
	\\ \\
	
	Du hörst während dem Spiel nur, ob Mannschaft 1 oder 2 ein Tor geschossen hat. Das Spiel hört auf, sobald du den Endpfiff hörst (in diesem Fall eine 0). Dann sollst du das Ergebnis angeben.
	\\ \\
	
	\subsection{Voraussetzungen} Um diese Aufgabe erfolgreich lösen zu können solltest du dich mit \code{if}-Bedingungen auskennen.
	
	\newpage
	
	\subsection{Beispiele}
	\subsubsection{Beispiel 1}
	Eingabe: \\ \\
	1 \\
	1 \\
	2 \\
	0
	\\ \\
	Du gibst auf der Konsole aus: \\ \\
	2-1
	
	\subsubsection{Beispiel 2}
	Eingabe: \\ \\
	2 \\
	1 \\
	1 \\
	2 \\
	2 \\
	2 \\
	0
	\\ \\
	Du gibst auf der Konsole aus: \\ \\
	2-4
	
\end{document}